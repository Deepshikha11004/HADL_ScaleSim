\documentclass[12pt]{article}

\usepackage[english]{babel}

% Set page size and margins
% Replace `letterpaper' with `a4paper' for UK/EU standard size
\usepackage[letterpaper,top=2cm,bottom=2cm,left=3cm,right=3cm,marginparwidth=1.75cm]{geometry}

% Useful packages
\usepackage{float}
\usepackage{placeins}
\usepackage{amsmath}
\usepackage{graphicx}
\usepackage{hyperref}
\hypersetup{
    colorlinks,
    citecolor=black,
    filecolor=black,
    linkcolor=black,
    urlcolor=black
}

\title{
{\Huge{ Assign  2
\\
HADL}}
}
\date{}
\begin{document}



\author{
Deepshikha  CS21BTECH11016
}
\maketitle


\section{TODO}
Why 1st layer less than 100?

\section{Summary of SCALE-SIM}

\subsection{What it does}

\begin{itemize}
    \item It is a cycle-accurate simulator for DNN accelerators.
    \item It takes in the CNN architecture and the accelerator configuration as input and gives the performance metrics as output.
    \item It computes performance, on-chip and off-chip memory accesss, and interface bandwidth.
    \item It can implement both scale-up and scale-out instances.

\end{itemize}

\subsection{How it does}
\begin{itemize}
    \item SCALE-SIM generates a cycle-accurate trace of the accelerator execution ,genearting an output which contains SRAM writes.
    \item THE SRAM trace shows the data movement and computation in the accelerator.
    \item The requests to SRAM  are the DRAM traces , which are used to estimate the interface bandwidth for given CNN.
\end{itemize}


\section{Configs}


\subsection{Eyeriss}
Array size : 12 x 14
\subsection{Google}
Array size : 256 x 256
\subsection{Scale}
Array size : 32 x 32

% \begin{center}
%     \begin{tabular}{ |c|c |c| c | }
%     \hline
%   & Eyeriss & Google & Scale \\
%     \hline

%     \end{tabular}
% \end{center}

% \section{CNN architecture}

% \subsection{MobileNet}


\section{Running CNN architecture on SCALE-SIM}

\subsection{Varying the configs}

\begin{enumerate}
    \item MobileNet
    \begin{enumerate}
        \item First layer :
        \begin{enumerate}
            \item The first conv layer has 224 x 224 x 3 size IFMAP.
            \item Scale config has 32 x 32 array size , so all of it is utilsed  as 224/32 = 7.
            \item Google config has 256 x 256 array size , it accomodates 224 x 224 x 3 IFMAP , but the rest of the array is wasted and so
            \item Eyeriss config has 12 x 14 array size , ao it cannot accomodate the entire IFMAP, so 224/12 = 18.66 and 224/14 = 16 , as 224 is not divisible by 12 , so rest of the array is wasted.
        \end{enumerate}

        \item Second Layer :
        \begin{enumerate}
            \item This is a pooling layer
            \item Pooling invloves accessing data in a non-regular pattern, which may not fully exploit the regular data access patterns.Thats why,mapping efficiency for all pooling layersis lesser compared to conv layers.
            \item Eyeriss gives better mapping efficiency than scale because of 112 is divisible by 14.
        \end{enumerate}

        \item Remaining layers :
        \begin{itemize}
            \item We see that for google config 
            \begin{itemize}
                \item When the number of channels are >= 256, mapping efficiency is 100 percent. This is due to the fact that the array size is 256 x 256.
                \item As the size of IFMAP is reduced, the mapping efficiency is reduced.
            \end{itemize} 
        \end{itemize}
    \end{enumerate}

    \item Resnet18
        \begin{itemize}
            \item We see it has no pooling layers, so the mapping efficiency always good.
            \item In google config 
                \begin{itemize}
                    \item in conv layer 3{\textunderscore}s , filter size decreases to 1 x 1, here we see efficiency reduces.
                    \item as number of filters increase as we saw in `mobilenet' (>=265 ), mapping efficiency increases to 100 percent.
                \end{itemize}
            \item 
        \end{itemize}


    \item GoogleNet
    \begin{itemize}
        

\end{enumerate}

\subsection*{Varying the architecture}

\begin{itemize}
    \item Eyeriss
        \begin{itemize}
            \item  We see that it gives beter efficiency for resnet18 than mobilenet. The difference is of filters.resnet18 in first layer has 7 x 7 filters while mobilenet has 3 x 3 filters , stride being same for both (2). Since more filter size means more data reuse/parallelism, resnet18 has better efficiency.
        \end{itemize}
\end{itemize}



\end{document}